% \begin{center}
% {\Large{\bf Project Summary}}\\*[3mm]
% {\bf Community Facility Support: The Community Surface Dynamics Modeling System (CSDMS)} \\*[3mm]
% \end{center}

\subsection*{Overview}

As the sciences that probe Earth's changing surface become more quantitative and prediction-oriented, they increasingly rely on computational modeling and model-data integration. Yet research progress has been impeded by the fragmented and inefficient nature of the supporting cyberinfrastructure. This project develops OpenEarthscape: an integrated suite of community-developed cyber resources for simulation and model-data integration, focusing on nine high-priority science frontiers: geohazards, water, basins, geologic records, topographic change, seafloor evolution, Arctic environments, the critical zone, and coastal ecosystems. Recent advances, including many derived from NSF investments, bring new opportunities for greater interoperability, efficiency, and reproducibility. OpenEarthscape will raise the proportion and quality of community-built software that adheres to the FAIR principles (Findable, Accessible, Interoperable, Reusable). Products and activities include EarthscapeHub: a JupyterHub server providing easy access to OpenEarthscape models, tools, and libraries; new capacity for creating and sharing reproducible model-data analysis packages; and major enhancements to the current Landlab and PyMT libraries for model construction and coupling. OpenEarthscape catalyzes efficiency by building new technology to improve numerical performance and developing an extended version of the well-adopted Basic Model Interface standard to address parallel architecture and coupling. OpenEarthscape fosters research productivity with improved library capabilities for data I/O and visualization, and with community resources for efficient software distribution and cross-platform compatibility.

\subsection*{Intellectual Merit}

To ensure close community engagement and collaborative development, OpenEarthscape builds on a set of demonstration projects, in partnership with community members, to collectively address the 9 science questions above. Primary projects are led by collaborating partners, and include marine sediment and carbon transport; post-wildfire landsliding and sediment impacts; flood inundation modeling; Arctic landscape change; and long-term terrain and basin evolution. Secondary projects are enabled by summer support for visiting graduate students. Student visitors learn and apply cybertools in focused projects such as creating a fully scripted and reproducible analysis package to accompany a journal article; or applying best-practice to their own code to reach a standard suitable for publication in an open-source software journal. Inspired by findings from science gateways, OpenEarthscape will use a variety of metrics. These will address facilitation of research (e.g., 3rd-party publications/products), resource adoption, community contributions, and early-career perspectives (e.g., exit interviews with summer students). The sustainability plan includes outreach to early-career scholars, and facilitation of community ownership of products by establishing volunteer community governance bodies around specific products, such as the evolving BMI standard.


\subsection*{Broader Impacts}

Broader impacts of the proposed project include partnership with undergraduate research programs that support traditionally underrepresented student populations, with the project team contributing introductory training in scientific computing---a marketable skill within and beyond geosciences. A novel educational element is the OpenEarthscape Simulator: a web-hosted visual simulation of a micro-continent evolving in response to pseudo-random climatic, volcanic, tectonic, and other events. The simulator provides students and the general public with an intriguing visualization of Earthscape dynamics, and provides a template for the research community to identify major defects in our current understanding. In addition, the project would maintain, support, and build on cyber-products originally developed for the Community Surface Dynamics Modeling System (CSDMS 3.0; EAR-1831623, 2018--2021). A further broad impact is the portability of key cyber concepts and tools beyond the core science communities within which they are developed.

Keywords: Model-data integration, FAIR software, CloudAccess, Community-partnered projects

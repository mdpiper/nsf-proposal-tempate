\section*{Data Management Plan}
\nonumber
%\(2 page limit)

\subsection*{Roles and responsibilities}
The data management activities during the project period will be performed in adherence with NSF policies and, where applicable, policies of the five collaborating institutions. The investigators will coordinate on the execution of this data management plan.

\subsection*{Data Definition}

This project will develop and maintain open-source licensed, freely available source code for tools, templates, tutorial notebooks, libraries, numerical models, and other digital artifacts in online repositories. The OpenEarthscape software and resource stack will be maintained in online repositories. For the purpose of this management plan, software products, codes, and digital educational resources are all defined here as ``data.'' Software and related products (including supporting items in text-based format, such as JSON-based Jupyter notebooks) will be maintained in git-based version-control repositories and hosted as MIT-licensed open-source products on GitHub. OpenEarthscape models and tools will include standardized metadata, with fields for authorship, provenance, version, properties of model simulation inputs and outputs (if applicable), and bibliographic parameters, such that the community can discover and understand user limitations. These metadata generally follow the Dublin Core metadata standard. A version-stamped installation of software tools and associated model and data components will be installed on the OpenEarthscapeHub server and updated as needed. Workflows and related materials generated by the project will be hosted in git-based repositories.

\subsection*{Period of Data Retention}

Data used in this project will generally be developed openly in public repositories, and ``released'' (with ``release'' here meaning a formal announcement of availability and capabilities) ahead of outreach events and/or in periodic communications such as newsletters. The data will be publicly available for at least a three-year period (or the project duration, if longer), in adherence with NSF Policies. 

\subsection*{Metadata for Models and Tools}

For model software contributed by partnering community members, metadata will be collected through online webforms via the CSDMS Model Repository, and become available online upon submission. These model metadata are stored through the community content management system of CSDMS, which is based on mediawiki. Two external plugins, Semantic Mediawiki \citep{krotzsch2006semantic}
and Cargo \citep{koren2018cargo}, 
allow mediawiki to act as a flexible knowledge management system that can also be populated and queried through application programming interfaces (APIs).

\subsection*{Data Sharing}

To facilitate discovery of OpenEarthscape software and other resources, they will be catalogued on the CSDMS web portal, which will host top-level documentation with links to individual product documentation and GitHub repositories. We envision a dedicated top-level item on the splash page that leads to OpenEarthscape. This approach gives visibility to OpenEarthscape, and provides multiple avenues of discovery (CSDMS portal, GitHub pages, and product documentation pages).


\subsection*{Model I/O Standards}

OpenEarthscape will use NetCDF as model input and output standard. NetCDF is a widely used file format that has been specifically designed for storing scientific data. NetCDF is a flexible, self-describing format that bundles data together with any relevant metadata. It can be used to store virtually any kind of data including spatial grids, time-indexed grid stacks, time series, time-indexed profiles and time indexed ``cube series.'' When appropriate, data are stored efficiently and compactly in binary form with transparent byte-order conversion. Due to the flexibility of the format, several standards have emerged that specify protocols for handling issues such as unstructured grids, names of physical quantities, inclusion of units, and so on; for these cases, OpenEarthscape will use the CF convention \citep{cf2017conventions}. Tools for data access will take advantage of the \texttt{xarray} package, which has a close link to NetCDF data structures and I/O.

\subsection*{Data Access and Reuse}

%CSDMS has a variety of data holdings and links to other data holdings on its wiki website at csdms.colorado.edu. CSDMS maintains a THREDDS Data Server that simplifies sharing netCDF files using OpenDAP.  %\citep{unidata2017thredds}. 
%These will be made available through Web Map Services. 
OpenEarthscape models, data, and documentation will be available under open-source licenses. Tutorials and training materials created for this project will be made available under Creative Commons license with attribution, and distributed via GitHub repositories (for ongoing active development) and via community sites such as HydroShare.



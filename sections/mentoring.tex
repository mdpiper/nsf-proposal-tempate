\section*{Postdoctoral Researcher Mentoring Plans}
%\ (1 page limit)

% \subsection*{Postdoctoral Researcher Mentoring Plan}

This Postdoctoral Researcher Mentoring plan establishes guidelines for work to be performed by the two Postdoctoral Researchers (PDR) in support of OpenEarthscape. One PDR, based at the University of Colorado, will be responsible for producing use-case demonstrations that illustrate how OpenEarthscape technology can be applied in the scientific domains represented by the project. The other PDR, based at Tulane University, will be responsible for building a source-to-sink modeling framework, using OpenEarthscape tools, as part of the Source-to-Sink demonstration project.

% We propose to employ one PDF to be jointly supervised between CSDMS and individual PIs in the community at large. Over year 2-5 of the proposed project period, CSDMS will solicit concise research proposals to advance earth surface process modeling science. Project teams are to consist of a PDF with a faculty mentor at an academic institution in the US. CSDMS will offer 50\% funding for 2 years, whereas the partnering faculty mentor needs to match the other 50\% for those years. A support letter of each applicant’s mentor ensures that this mentoring plan is agreed upon. The CSDMS Executive Committee as an independent entity will review proposals for intellectual merit, novelty of research, match with current community science themes, and fit with CSDMS science teams priorities. 

The postdoctoral researchers will be provided with a unique training opportunity: they will participate in the OpenEarthscape annual meeting, be part of cyber-skill building sessions through webinars, and participate in the Earth Surface Processes Summer Institute (ESPIn). The project will allow them to apply these skills and their familiarity with cyberframework tools to make substantial science advances on topics of interest as developed jointly with independent advisors from the larger community through the OpenEarthscape Computational Summer Science Program. The PDRs will come away from their postdoctoral research projects with a broader skill set in earth surface and environmental computation. The University of Colorado PDR will work primarily with a faculty member (PI G. Tucker) and a research professional mentor (senior software engineer and Co-PI E. Hutton)  The Tulane PDR will work primarily with Tulane faculty member and Co-PI N. Gasparini.

Essential elements of PDR mentoring include:
\begin{compactenum}
\item Orientation, including in-depth conversations between PDR, and OpenEarthscape mentors. Expectations will be discussed and agreed upon in advance. Orientation topics include: a) the amount of independence the PDR requires, b) interaction with research group members, c) productivity, including publications, d) work habits.

\item Career Counseling will be directed at providing the PDR: skills, knowledge, and experience needed to excel in his/her career.  Mentors will engage in career planning, by pointing out job opportunities and helping to evaluate advertisements and application materials.

\item Proposal Writing Experience will be gained by involvement of the PDR in proposals prepared by their mentors. The PDR will have an opportunity to learn best practices in proposal preparation including identification of key research questions, definition of objectives, description of approach and rationale, and construction of a work plan, timeline, and budget.

\item Publications and Presentations are expected from their work. These will be prepared under the direction of the PDR and in collaboration with their mentors. Attendance of the OpenEarthscape Annual Meeting is expected. The PDR will receive guidance and training in the preparation of manuscripts for scientific journals and presentations at conferences. 

\item Teaching and Mentoring Skills will be developed in the context of regular meetings within their local research group. The PDR will lead webinars to share across the OpenEarthscape community.

\item Instruction in Professional Practices will be provided by attending the 10-day Earth Surface Processes Summer Institute, with emphasis on modeling, reproducibility of modeling projects, software development, high-performance computing, and uncertainty and sensitivity testing.

% \item Success of the Mentoring Plan will be assessed by biennial interviews by a volunteer chair of one of the Working Groups or Focus Research groups who is not the supervisor of the postdoctoral researcher.
\end{compactenum}

% \subsection*{Postdoctoral Researcher Mentoring Plan (Gasparini)}

% Gasparini will supervise a postdoctoral researcher at Tulane University. The mentoring plan discussed here is designed to create a productive work environment for the postdoctoral researcher (PDR) so that they can both excel in their research while at Tulane and also gain the skills and experience needed to continue on their desired career path.%

% \begin{compactenum}

% \item Within the first week of employment or switching on to this project, the PDR and Gasparini will discuss and document expectations and goals for the postdoctoral researcher’s period of employment/this project. They will develop an Individual Development Plan (IDP) to define their research and career goals, identify gaps in training, develop expectations for communication with and input from the advisor, and set a realistic schedule for completion of publications and other research products. The PDR will be encouraged to use an on-line tool for IDPs developed by the AAAS (American Association for the Advancement of Science). This document will include both Gasparini’s and the PDR’s goals and expectations. Target deadlines will be set when appropriate.

% \item The document will be revisited at least once every six months to assess progress and reassess goals and target dates.

% \item The PDR will be encouraged to take advantage of on-campus resources for career planning and application support, such as those available through the Tulane Office of Graduate and Postdoctoral Studies.

% \item The PDR will be encouraged to attend workshops focused on career development and proposal writing. For example, proposal-writing workshops or public speaking workshops might be applicable, depending the PDR's career goals.

% \item Gasparini will also work with the PDR to help them prepare for job interviews and seminars, evaluate potential positions, and negotiate start-up packages.

% \item The PDR will be offered the opportunity to be a guest lecturer in a class period to present and discuss a topic falling under their expertise. This will be most useful if the PDR is interested in an academic career path, although teaching also improves public speaking skills.

% \item The PDR will participate in all reports and scientific publications related to the project. They will take the lead when appropriate.

% \item The PDR will participate in all virtual meetings related to this project.

% \item The PDR will be encouraged to present work at conferences beyond those specifically mentioned in the proposal. This includes presenting work from their research prior to coming to Tulane, as well as work from this project.

% \item The PDR will be given opportunities to contribute to proposal writing.

% \item The PDR will be encouraged to attend workshops on unconscious bias and inclusion, when held at Tulane or at national meetings.

% \item The PDR will be given the opportunity to work with and mentor Gasparini’s graduate and undergraduate students.

% \item The PDR will be encouraged to attend the weekly research seminars at Tulane in both the department of Earth and Environmental Sciences and the Center for Computational Sciences (CCS). They will also be encouraged to participate in activities through CSDMS, such as webinars and virtual trainings.

%\end{compactenum}
